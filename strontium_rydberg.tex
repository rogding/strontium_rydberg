\documentclass{article}

\usepackage{amsmath}
\usepackage{amssymb}
\usepackage{mathtools}
\usepackage{multirow}
\usepackage{float}
\usepackage{graphicx}
\usepackage{braket}
\usepackage{hyperref}
\usepackage{color}
\usepackage[]{siunitx}
\usepackage[style=phys,
			articletitle=false,
			biblabel=brackets,
			chaptertitle=false,
			pageranges=false]{biblatex}

% Additional options
\addbibresource{bibliography.bib}
\DeclareSIUnit\Torr{Torr}

% Macros
\newcommand{\tsup}{\textsuperscript}													% superscript outside math mode.
\newcommand{\tsub}{\textsubscript}														% subscript outside math mode.
\newcommand{\Sr}[1]{\tsup{#1}\textnormal{Sr}}											% {}^{xx}Sr outside math mode.
\newcommand{\Srion}[1]{\textnormal{Sr}\tsup{#1}}										% {}Sr^{xx} outside math mode.
\newcommand{\Li}[1]{\tsup{#1}\textnormal{Li}}											% {}^{xx}Li outside math mode.
\newcommand{\SLJ}[3]{\tsup{#1}\textnormal{#2}\tsub{#3}}
\newcommand{\SLJF}[4]{\tsup{#1}\textnormal{#2}\tsub{#3}\textnormal{, }{F = {#4}}}
\newcommand{\nSLJ}[4]{{#1}\textnormal{ }\SLJ{#2}{#3}{#4}}
\newcommand{\nSLJm}[5]{{#1}\textnormal{ }\SLJ{#2}{#3}{#4}\textnormal{, }{m_J = {#5}}}
\newcommand{\nSLJF}[5]{{#1}\textnormal{ }\SLJ{#2}{#3}{#4}\textnormal{, }{F = {#5}}}

% Front matter
\title{Notes on the strontium rydberg data}
\author{Roger Ding}
%\date
 
% Start of the document.
\begin{document}

\maketitle


Some notes on the transcribed data.

\section{\fullcite{esh_1977}}

\subsection{Changes to source material}

\subsubsection{Observed energy of $\nSLJ{5s6d}{3}{D}{2}$}

The calculated energy of the $\nSLJ{5s6d}{3}{D}{2}$ given in Table~IV (the ${J=2}$ states) as \SI{36691.11}{\per\cm} is likely a typo. I think it should be \SI{39691.11}{\per\cm} instead which is supported by \cite{san_2010} where the energy listed for this state is \SI{39690.802}{\per\cm} (i.e., a check that the observed energy is correct). Making this change means the error between observed and calculated energies is now \num{-0.260} and agrees with the error in Table~IV.

\subsubsection{Assignments for ${n=15,16}$ ${\SLJ{1}{D}{2}}$ and ${\SLJ{3}{D}{2}}$ lines}

\citeauthor{esh_1977} doesn't definitively assign ${n=15,16}$ ${\nSLJ{5snd}{1}{D}{2}}$ and ${\nSLJ{5snd}{3}{D}{2}}$ states, instead labeling those four lines as ${\SLJ{1,3}{D}{2}}$.

From \cite{san_2010}, the energy of the ${\nSLJ{5s15d}{3}{D}{2}}$ (${\nSLJ{5s15d}{1}{D}{2}}$) is given as \SI{45276.65}{\per\cm} (\SI{45263.6196}{\per\cm}). Noting that the $\SLJ{1}{D}{2}$ is lower in energy than the $\SLJ{3}{D}{2}$ state, I label the lower (higher) energy state as $\SLJ{1}{D}{2}$ ($\SLJ{3}{D}{2}$).

Similarly for the $5s16d$ states, \cite{san_2010} gives the energy of the $\nSLJ{5s16d}{1}{D}{2}$ ($\nSLJ{5s16d}{3}{D}{2}$) to be \SI{45362.1272}{\per\cm} (\SI{45350.35}{\per\cm}). In this case, the $\SLJ{1}{D}{2}$ is now higher in energy than the $\SLJ{3}{D}{2}$ state so I label them accordingly in my table.

\section{\fullcite{ewp_1976}}

The authors state that they estimate their final uncertainties to be \SI{\pm 0.3}{\per\cm}. 

In Table~1, the authors do not appear to assign $n$-values for $\nSLJ{5snd}{3}{D}{2}$ lines, leaving that part of the table blank. In my data table, I left unassigned or unidentified lines out.

\section{\fullcite{gac_1968}}

In this particular paper, they do not give an uncertainty but in an earlier paper (\cite{gaw_1966}), Garton states their error to be not greater than \SI{0.008}{\angstrom} so I took that to be the uncertainty for their values in the present paper. This also suggests that their measured units are wavelength and not wavenumbers.

I'll need to look into why Table~1 ($\nSLJ{5snp}{1}{P}{1}$) has wavelengths to three decimal places whereas Table~2 ($\nSLJ{5snp}{3}{P}{1}$) only gives wavelengths to two decimals. 

\section{\fullcite{rub_1978}}

This paper measures state energies NOT relative to the ground state. 

They measured wavelengths as their units and gaven an uncertainty of \SI{0.002}{\nm}. 

When I write the analysis code for analyzing the data, I think I'll use the most recent (and hopefully more accurate) measurements of the initial states they're exciting from and then using the measured wavelength and uncertainty to calculate the term energies. Actually, I'll need to figure out how they converted from the wavelength in air to vacuum wavenumber (i.e., what formula they used to account for humidity, temperature, etc.). 

\subsubsection{Notes}

The ``Previous'' column gives the last few digits from the cited source for comparison with this paper's values. 

\section{\fullcite{awe_1979}}

(Placeholder)

\section*{\citetitle{blt_1982a}}

They don't explicitly specify which isotope they measure the term energies for but it appears to be for \Sr{88}. They used two-photon spectroscopy (retro-reflected blue laser) on strontium heated to \SI{650}{\celsius} (corresponding to \SI{25}{\milli\Torr}). 

\section*{\citetitle{bls_1982a}}
 
\printbibliography

\end{document}