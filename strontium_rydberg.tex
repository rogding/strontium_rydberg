\documentclass{article}

\usepackage{amsmath}
\usepackage{amssymb}
\usepackage{mathtools}
\usepackage{multirow}
\usepackage{float}
\usepackage{graphicx}
\usepackage{braket}
\usepackage{hyperref}
\usepackage{color}
\usepackage[]{siunitx}
\usepackage[style=phys,
			articletitle=false,
			biblabel=brackets,
			chaptertitle=false,
			pageranges=false]{biblatex}

% Additional options
\addbibresource{bibliography.bib}
\DeclareSIUnit\Torr{Torr}

% Macros
\newcommand{\tsup}{\textsuperscript}													% superscript outside math mode.
\newcommand{\tsub}{\textsubscript}														% subscript outside math mode.
\newcommand{\Sr}[1]{\tsup{#1}\textnormal{Sr}}											% {}^{xx}Sr outside math mode.
\newcommand{\Srion}[1]{\textnormal{Sr}\tsup{#1}}										% {}Sr^{xx} outside math mode.
\newcommand{\Li}[1]{\tsup{#1}\textnormal{Li}}											% {}^{xx}Li outside math mode.
\newcommand{\SLJ}[3]{\tsup{#1}\textnormal{#2}\tsub{#3}}
\newcommand{\SLJF}[4]{\tsup{#1}\textnormal{#2}\tsub{#3}\textnormal{, }{F = {#4}}}
\newcommand{\nSLJ}[4]{{#1}\textnormal{ }\SLJ{#2}{#3}{#4}}
\newcommand{\nSLJm}[5]{{#1}\textnormal{ }\SLJ{#2}{#3}{#4}\textnormal{, }{m_J = {#5}}}
\newcommand{\nSLJF}[5]{{#1}\textnormal{ }\SLJ{#2}{#3}{#4}\textnormal{, }{F = {#5}}}

% Front matter
\title{Notes on the strontium rydberg data}
\author{Roger Ding}
%\date
 
% Start of the document.
\begin{document}

\maketitle


Some notes on the transcribed data.

\section{\fullcite{esh_1977}}

\subsection{Changes}

\subsubsection{Observed energy of $\nSLJ{5s6d}{3}{D}{2}$}

The calculated energy of the $\nSLJ{5s6d}{3}{D}{2}$ given in Table~IV (the ${J=2}$ states) as \SI{36691.11}{\per\cm} is likely a typo. I think it should be \SI{39691.11}{\per\cm} instead which is supported by \cite{san_2010} where the energy listed for this state is \SI{39690.802}{\per\cm} (i.e., a check that the observed energy is correct). Making this change means the error between observed and calculated energies is now \num{-0.260} and agrees with the error in Table~IV.

\subsubsection{Assignments for ${n=15,16}$ ${\SLJ{1}{D}{2}}$ and ${\SLJ{3}{D}{2}}$ lines}

\citeauthor{esh_1977} doesn't definitively assign ${n=15,16}$ ${\nSLJ{5snd}{1}{D}{2}}$ and ${\nSLJ{5snd}{3}{D}{2}}$ states, instead labeling those four lines as ${\SLJ{1,3}{D}{2}}$.

From \cite{san_2010}, the energy of the ${\nSLJ{5s15d}{3}{D}{2}}$ (${\nSLJ{5s15d}{1}{D}{2}}$) is given as \SI{45276.65}{\per\cm} (\SI{45263.6196}{\per\cm}). Noting that the $\SLJ{1}{D}{2}$ is lower in energy than the $\SLJ{3}{D}{2}$ state, I label the lower (higher) energy state as $\SLJ{1}{D}{2}$ ($\SLJ{3}{D}{2}$).

Similarly for the $5s16d$ states, \cite{san_2010} gives the energy of the $\nSLJ{5s16d}{1}{D}{2}$ ($\nSLJ{5s16d}{3}{D}{2}$) to be \SI{45362.1272}{\per\cm} (\SI{45350.35}{\per\cm}). In this case, the $\SLJ{1}{D}{2}$ is now higher in energy than the $\SLJ{3}{D}{2}$ state so I label them accordingly in my table.

\section{\citetitle{blt_1982a}}

They don't explicity specify which isotope they measure the term energies for but it appears to be for \Sr{88}. They used two-photon spectroscopy (retro-reflected blue laser) on strontium heated to \SI{650}{\celsius} (corresponding to \SI{25}{\milli\Torr}). 

\section{\citetitle{esh_1977a}}

\section{\citetitle{rub_1978a}}

\section{\citetitle{awe_1979a}}

\section{\citetitle{bls_1982a}}
 
\printbibliography

\end{document}